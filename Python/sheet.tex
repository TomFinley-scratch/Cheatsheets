\documentclass[landscape,10pt,letterpaper]{article}

\usepackage{epsfig}
\usepackage{graphics}
\usepackage{graphicx}
\usepackage{amsfonts}
\usepackage{algorithmic}
\usepackage{enumitem}
%\usepackage[paperwidth=8.5in,paperheight=11in]{geometry}
\usepackage[paperwidth=12.75in,paperheight=16.5in]{geometry}
\usepackage{multicol}
\usepackage{textcomp}
\usepackage{ifthen}
\usepackage{aeguill}
\usepackage{color}
\usepackage{verbatim}

\textwidth = 10.6 in
\textheight = 7.9 in
\oddsidemargin = -0.73 in
\evensidemargin = -0.73 in
\topmargin = -1.2 in

\textwidth = 1.5 \textwidth
\textheight = 1.5 \textheight

\newcounter{pyversion}
\setcounter{pyversion}{27}

\pagestyle{empty}
\newcommand{\columnseprulecolor}{\color{red}}

\newcommand{\argmax}{\mathop{\mbox{argmax}}}
\newcommand{\argmin}{\mathop{\mbox{argmin}}}
\newcommand{\heading}[1]{\vspace{-1.5em} \section*{\rule{.5em}{.5em} #1} \vspace{-1.0em}}
\newcommand{\subheading}[1]{\vspace{-1.2em} \subsection*{#1} \vspace{-0.8em}}

\newcommand{\eq}{\ensuremath{\mbox{\textdblhyphen\textdblhyphen}}}

% Py function or type.
\newcommand{\pyf}[1]{\ensuremath{\mathop{\mathtt{#1}}}}
% Py version filter, include only if pyversion at least this...
\newcommand{\pyv}[2]{\ifthenelse{\value{pyversion}<#1}{}{#2}}
% Py version filter, include only if pyversion is less than this...
\newcommand{\pyV}[2]{\ifthenelse{\value{pyversion}<#1}{#2}{}}
% Py string.
\newcommand{\pys}[1]{\ensuremath{\mathtt{#1}}}
% Py keyword.
\newcommand{\pyk}[1]{\ensuremath{\mathtt{#1}}}
% Py type.
\newcommand{\pyt}[1]{\pyf{#1}}
% Py package.
\newcommand{\pyp}[1]{\pyf{#1}}
% Putting stuff in __foo__
\newcommand{\pysec}[1]{\pyf{\_\_{#1}\_\_}}
% Optional brackets.
\newcommand{\optional}[1]{{[}{#1}{]}}
% For a heading in a tabbing environment.
\newcommand{\thead}[1]{\` {\large \textbf{#1}} \\}

\newcommand{\class}{\ensuremath{\mathbb{C}}}
\newcommand{\insta}{\ensuremath{\mathbb{I}}}
\newcommand{\instb}{\ensuremath{\mathbb{J}}}

\newcommand{\Slash}{\hspace{-.25em}/\hspace{-.25em}}

\begin{document}

\begin{multicols}{6}

Thomas Finley, \texttt{tfinley@gmail.com}, \texttt{http://tfinley.net/}

\heading{Built-in Types}
\vspace{1.0em}
%Any type $\pyt{t}$ instantiated with no arguments

%Types $\pyt{t}$ callable as $\pyt{t}()$ to return default (empty, zero) instance.

\subheading{Numerics {\small \pyt{int}, \pyt{float}, \pyt{long}, \pyt{complex}}}

\begin{tabbing}
\hspace{2em}\= \kill
\' $\pyf{int}(),\pyf{float}(),\pyf{long}(),\pyf{comple}()$ \` zero \\
\' $\pyf{int}(x),\pyf{float}(x),\pyf{long}(x),\pyf{complex}(x)$ \\ \` parse string or cast number \\
\' $\pyf{int}(x, b), \pyf{long}(x, b)$ \` parse $b$-base string $x$ \\
\' $\pyf{complex}(\optional{a, \optional{b}})$ \` complex number $a+bj$
\end{tabbing} 
\begin{tabbing}
\hspace{2em}\= \kill
$n.\pyf{conjugate()}$ \` complex conjugate \\
$n.\pys{numerator}$ \` the $a$ in $\frac{a}{b}$ \\
$n.\pys{denominator}$ \` the $b$ in $\frac{a}{b}$ \\
$n.\pys{real}$ \` the $a$ in $a+bj$ \\
$n.\pys{imag}$ \` the $b$ in $a+bj$
\pyv{27}{\\ \'$n.\pyf{bit\_length}()$ \` bits to hold $n$ (int/long)}
\pyv{26}{\\ \'$n.\pyf{as\_integer\_ratio}()$ \` $(a,b)$ with $n=\frac{a}{b}$}
\end{tabbing}
Operators: the uniquitous arithmetic \verb.+,-,*,/,\%., bitwise \verb.|,^,&,<<,>>,~., as well as \verb.//. (floored quotient) and \verb.**. (power).

\subheading{Booleans \pyt{bool}}

Booleans can be treated like numbers, where $\pys{False}\eq 0$ and $\pys{True}\eq 1$.

\begin{tabbing}
\hspace{2em}\= \kill
\'$\pyf{bool}(\optional{x})$ \` \pys{False} iff $x\eq \pys{None}\!,\pys{False}\!,0\!,$empty
\end{tabbing}

\subheading{Mutable Sequences}

\begin{tabbing}
\hspace{2em}\= \kill
\'$\pyf{append}(x)$ \` add single item $x$ to end \\
\'$\pyf{extend}(it)$ \` add $it$'s items to end \\
\'$\pyf{count}(x)$ \` number of items equal to $x$ \\
\'$\pyf{index}(x,\!\optional{i,\!\optional{j}})$ \` lowest index $k\!\in\![i,j)$ for $x$ \\
\'$\pyf{insert}(i, x)$ \` insert so item at index $i$ is $x$ \\
\'$\pyf{pop}(\optional{i})$ \` remove, return last item or item $i$ \\
\'$\pyf{remove}(x)$ \` delete first element equal to $x$ \\
\'$\pyf{reverse}()$ \` reverse sequence in place \\
\'$\pyf{sort}(\optional{cmp, \optional{key, \optional{reverse}}})$ \` sort in place
\end{tabbing} \vspace{-1em}
Both \pyf{index} and \pyf{remove} raise \pys{ValueError} if element unfound.

\subheading{Strings \pyt{str}, \pyt{unicode}}

Both types derived from \pyt{basestring}.

Below, $st$ is the string instance.  For methods, when $b$, $e$ appear, assume the operation is limited to the substring from $b$ inclusive to $e$ exclusive.

\begin{tabbing}
\hspace{2em}\=\hspace{1em}\= \kill
\'$\pyf{capitalize}()$ \` only first char uppercase \\
\'$\pyf{center}(w, \optional{fchar})$ \` $f\!char$ padded to $w$ len \\
\'$\pyf{count}(s, \optional{b, \optional{e}})$ \` num of non-overlap $s$ \\
\'$\pyf{decode}(\optional{enc, \optional{err}})$ \` decode encoded string \\
\'$\pyf{encode}(\optional{enc, \optional{err}})$ \` encode string \\
\> \pys{enc} \` often \verb.utf8.; see codecs \\
\> \pys{err} \` \verb.strict., raise \pyt{UnicodeError} \\
\` \verb.ignore., \verb.xmlcharrefreplace. \\
\` \verb.replace., \verb.backslashreplace. \\
\'$\pyf{endswith}(s, \optional{b, \optional{e}})$ \` suffix is $s$ \\ %\pyv{25}{or one of $s$ for tuple} \\
%\'$\pyf{endswith}(s, \optional{b, \optional{e}})$ \`   (py2.5) \\
\'$\pyf{expandtabs}(\optional{tabsize})$ \` subs tab w/spaces \\
\'$\pyf{find}(s, \optional{b, \optional{e}})$ \` position of $s$ in $st$, or $-1$ \\
\'$\pyf{format}(*a, *\!*\!kw)$ \` new \{\} style formatting \\
\'$\pyf{index}(\!s,\!\optional{b,\!\optional{e}}\!)$ \` if unfound raise \pys{ValueError} \\
\'$\pyf{isalnum}()$ \` consists of alphanumerics \\
\'$\pyf{isalpha}()$ \` consists of alphabetics \\
\'$\pyf{isdigit}()$ \` consists of digits \\
\'$\pyf{islower}()$ \` has alphabetics, all lowercase \\
\'$\pyf{isspace}()$ \` consists of whitespace \\
\'$\pyf{istitle}()$ \` has alphabetics, titlecased \\
\'$\pyf{isupper}()$ \` has alphabetics, all uppercase \\
\'$\pyf{join}(it)$ \` $it$'s items delimited with string \\
\'$\pyf{ljust}(w, \optional{f\!char})$ \` right-padded to $w$-len \\
\'$\pyf{lower}()$ \` lowercase \\
\'$\pyf{lstrip}(\optional{chars})$ \` remove chars from start \\
\pyv{25}{\'$\pyf{partition}(sep)$ \` $(pre, sep, su\!f\!)$ or $(st,\!\pys{''},\!\pys{''})$ \\}
\'$\pyf{replace}(old, new, \optional{c})$ \` replace all or $c$ $old$ \\
\'$\pyf{rfind}(s, \optional{b, \optional{e}})$ \` like \pyf{find} but from end \\
\'$\pyf{rindex}(s, \optional{b, \optional{e}})$ \` like \pyf{index} but from end \\
\'$\pyf{rjust}(w, \optional{f\!char})$ \` left-padded to $w$-len \\
\pyv{25}{\'$\pyf{rpartition}(sep)$ \` like \pyf{partit\!i\!o\!n} from end \\}
\pyv{24}{\'$\pyf{rsplit}(\optional{s, \optional{m}})$ \` like \pyf{split} but from end \\}
\'$\pyf{rstrip}(\optional{cs})$ \` remove chars from end \\
\'$\pyf{split}(\optional{s, \optional{m}})$ \` $s$-delimited substrings \\
\> $s$ \` if \pys{None}, whitesp. delim., no \pys{''} result \\
\> $m$ \` get $m$ substrings, plus remainder \\
\'$\pyf{splitlines}(\optional{keepends})$ \` list of lines,  \\
\'$\pyf{startswith}(p, \optional{b, \optional{e}})$ \` prefix is $p$ \\
\'$\pyf{strip}(\optional{cs})$ \` remove chars from start/end \\
\'$\pyf{swapcase}()$ \` lower to upper, upper to lower \\
\'$\pyf{title}()$ \` titlecased (first letters upcased) \\
\'$\pyf{translate}(tbl, \optional{delchrs})$ \` who uses this?! \\
\'$\pyf{upper}()$ \` uppercase \\
\'$\pyf{zfill}(width)$ \` leftfills with \pys{0}, handles \pys{\pm} \\
\thead{\pyt{unicode}}
\'$\pyf{isnumeric}()$ \` unicode numeric test \\
\'$\pyf{isdecimal}()$ \` unciode decimal test
\end{tabbing}

\pyV{24}{\begin{comment}} % set/frozenset do not exist prior to Python 2.4, though they are in a library package "sets" as Set/ImmutableSet.
\subheading{Sets \pyt{set}, \pyt{frozenset}}

In this method table, $s$ is our set instance.  The input $t$ can be any iterable.  Editing methods (starting with \pyf{update}) exist for \pyt{set}, not \pyt{frozenset}.

\begin{tabbing}
\hspace{2em}\= \kill
\pyv{26}{\'$\pyf{isdisjoint}(t)$ \` if no item also in $t$ \\}
\'$\pyf{issubset}(t)$ \` if all items also in $t$ \\
\'$\pyf{issuperset}(t)$ \` if all of $t$'s items also in $s$ \\
\'$\pyf{union}(t, ...)$ \` set with the items in $t$ or $s$ \\
\'$\pyf{intersection}(t, ...)$ \` ...in both $t$ and $s$ \\
\'$\pyf{difference}(t, ...)$ \` ...in $s$ but not $t$ \\
\'$\pyf{symmetric\_difference}(\!t\!)$ \` either, not both \\
\'$\pyf{copy}()$ \` give shallow copy \\
\'$\pyf{update}(t, ...)$ \` add $t$'s items \\
\'$\pyf{intersection\_update}(t, ...)$ \\ \` keep only items also in $t$ \\
\'$\pyf{difference\_update}(t, ...)$ \\ \` discard any items also in $t$ \\
\'$\pyf{symmetric\_difference\_update}(t)$ \\ \` keep items in either set, but not both \\
\'$\pyf{add}(x)$ \` add $x$ to set \\
\'$\pyf{remove}(x)$ \` remove $x$, \pyt{KeyError} if $x \not \in s$ \\
\'$\pyf{discard}(x)$ \` same, doesn't raise \pyt{KeyError} \\
\'$\pyf{pop}()$ \` remove, return arbitrary item\\
\` raises \pyt{KeyError} if empty \\
\'$\pyf{clear}()$ \` remove all items
\end{tabbing}
When using operators, unlike the methods, $t$ must be another set.  Inplace operators (\verb.|=., \verb.&=., \verb.-=., \verb.^=.) exist for \pyt{set}s.
\begin{multicols}{3}
\begin{tabbing}
\hspace{2em}\= \kill
\' $s\verb.<=.t$ \` $s \subseteq t$ \\ 
\' $s\verb.>=.t$ \` $s \supseteq t$ \\ 
\' $s\verb.<.t$ \` $s \subset t$ \\ 
\' $s\verb.>.t$ \` $s \supset t$ \\ 
\' $s\verb.|.t$ \` $s \cup t$ \\ 
\' $s\verb.&.t$ \` $s \cap t$ \\ 
\' $s\verb.-.t$ \` $s \setminus t$ \\
\' $s\verb.^.t$ \` $s\!\cup\!t\!\setminus\!s\!\cap\!t$
\end{tabbing}
\end{multicols}
\pyV{24}{\end{comment}}

\subheading{Maps \pyt{dict}}

In this method table, $d$ is our map instance

\begin{tabbing}
\hspace{2em}\=\hspace{1em}\= \kill
\'$\pyf{iter}(d)$ \` iterator over keys \\
\'$\pyf{clear}()$ \` empties mapping \\
\'$\pyf{copy}()$ \` returns copy of map \\
\'$\pyf{\class.fromkeys}(seq, \optional{val})$ \` make map $d$ so \\
\` $d[seq[i]]\eq val[i]$, or $\eq \pys{None}$ \\
\'$\pyf{get}(k, \optional{de\!f})$ \` $d[k]$ if exists, or $de\!f$/\pys{None} \\
\'$\pyf{has\_key}(k)$ \` if $d[k]$ exists, $k\!\verb. in .\!d$ preferred \\
\'$\pyf{items}()$ \` list of key, value tuples \\
\'$\pyf{keys}()$ \` list of keys \\
\'$\pyf{values}()$ \` list of values \\
\'$\pyf{iteritems}()$, $\pyf{iterkeys}()$, $\pyf{itervalues}()$ \` \\
\` same, but iterators, doesn't copy a list \\
\pyv{27}{\'$\pyf{viewitems}()$, $\pyf{viewkeys}()$, $\pyf{viewvalues}()$ \` \\ \` same, but dynamic ``view'' objects \\}
\'$\pyf{pop}(k, \optional{de\!f})$ \` remove, return $d[k]$ (or $de\!f$) \\
\`  \pyt{KeyError} if not default, $k \verb. not in . d$ \\
\'$\pyf{popitem}()$ \` remove, return arb. key, value \\
\` raises \pyt{KeyError} if empty \\
\'$\pyf{setdefault}(k, \optional{de\!f})$ \` if $k\!\not \in\!d$, $d[k]=de\!f$ \\
\` returns $d[k]$ \\
\'$\pyf{update}(\optional{o})$ \` add $o$'s mappings
\end{tabbing}

\subheading{File Objects}
Aside from actual files, file-like objects implementing a subset of methods/members are commonly used in Python code:

\begin{tabbing}
\hspace{2em}\=\hspace{1em}\= \kill
\'$\pyf{close}()$ \` close the file \\
\'$\pyf{flush}()$ \` flush file's internal buffer \\
\'$\pyf{fileno}()$ \` integer file descriptor \\
\'$\pyf{isatty}()$ \` file connected to tty-like device \\
\'$\pyf{next}()$ \` next input line \\
\'$\pyf{read}(\optional{s})$ \` read $s$ bytes, or till EOF \\
\'$\pyf{readline}(\optional{s})$ \` read $s$ bytes, or till EOL \\
\'$\pyf{readlines}(\optional{s})$ \` read about $s$ byes of lines \\
%\'$\pyf{xreadlines}()$ \`  \\
\'$\pyf{seek}(o, \optional{w})$ \` set file position $o$ bytes from \\
\` start/current/end ($w=0/1/2$) \\
\'$\pyf{tell}()$ \` file's current position \\
\'$\pyf{truncate}(\optional{s})$ \` truncate to current pos \\
\'$\pyf{write}(str)$ \` write $str$ to file \\
\'$\pyf{writelines}(it)$ \` write strs (no \verb.\n. added) \\
\'$\pyf{closed}$ \` if file closed \\
\'$\pyf{encoding}$ \` encoding when writing \pyt{unicode} \\
\pyv{26}{\'$\pyf{errors}$ \` unicode error handler \\}
\'$\pyf{mode}$ \` I/O mode for file \\
\'$\pyf{name}$ \` filename \\
\'$\pyf{newlines}$ \` detected newline if \verb.U. in mode \\
\'$\pyf{softspace}$ \` next \verb.print. should add space
\end{tabbing}

\heading{Special Method Names}

Some behaviors (e.g., acting like number/sequence/map) implementable with special methods.  \emph{Every} method is prefixed/suffixed with \verb.__., and accept $sel\!f$ as first argument (except \verb.__new__.), so these are omitted, so $\pyt{foo}(a)$ is really $\pyt{\_\_foo\_\_}(sel\!f, a)$.

\begin{tabbing}
\hspace{2em}\= \kill
\'$\pyf{new}(cls, \optional{..})$ \` new instance of $cls$ (usually) \\
\'$\pyf{init}(\optional{..})$ \` initialize instance \\
\'$\pyf{del}()$ \` called when about to be destroyed \\
\'$\pyf{repr}()$ \` ``formal'' representation \\
\'$\pyf{str}()$ \` ``informal'' representation \\
\'$\pyf{lt}\!/\!\pyf{le}\!/\!\pyf{eq}\!/\!\pyf{ne}\!/\!\pyf{gt}\!/\!\pyf{ge}(other)$ \` rich compare \\
\'$\pyf{cmp}(other)$ \` alternative; neg if $sel\!f\!<\!other$ \\
\'$\pyf{hash}()$ \` integer used in hashes \\
\'$\pyf{nonzero}()$ \` if implemented, used in \pyf{bool()} \\
\'$\pyf{unicode}()$ \` should return unicode \\
\'$\pyf{getattr}(a)$ \` called \emph{only} if $o.a$ unfound \\
\'$\pyf{setattr}(a, v)$ \` called on $o.a=v$ attempt \\
\'$\pyf{delattr}(a)$ \` called on $\pys{del} o.a$ attempt \\
\'$\pyf{getattribute}(a)$ \` called on $o.a$ attempt \\
\thead{Emulating Functions}
\'$\pyf{call}(\optional{args})$ \` called when used as function \\
\thead{Emulating Containers}
\'$\pyf{len}()$ \` length of object \\
\'$\pyf{getitem}(k)$ \` $k$ is int/slice (seq), key (map) \\
\'$\pyf{setitem}(k,v)$ \` like $\pyt{getitem}$; sets $o[k]=v$ \\
\'$\pyf{delitem}(k)$ \` removes item at $k$ \\
\` should raise \pyt{Type/Index/KeyError} if $k$ \\
\` bad type/seq index/unfound map index \\
\'$\pyf{iter}()$ \` iterator on items (keys for map) \\
\'$\pyf{reversed}()$ \` same, but reverse iterator \\
\'$\pyf{contains}(item)$ \` supports \verb.in./\verb.not in. test \\
\thead{Emulating Numerics}
\'$\pyf{add}\Slash{}\pyf{sub}\Slash{}\pyf{mul}\Slash{}\pyf{div}\Slash{}\pyf{truediv}\Slash{}\pyf{floordiv}\Slash{}\pyf{mod}$\\
\'$\pyf{divmod}\Slash{}\pyf{pow}\Slash{}\pyf{lshift}\Slash{}\pyf{rshift}\Slash{}\pyf{and}\Slash{}\pyf{xor}\Slash{}\pyf{or}(\!b\!)$ \\
\`supports $a+b$, etc.; \pyt{NotImplemented} \\
\`raised for unsupported types \\
\'$\pyf{radd}\Slash{}\pyf{rsub}\Slash{}\cdots(b)$ \` supports $b+a$, etc. \\
\` called if $b$'s non-\verb.r. operator inapplicable \\
\'$\pyf{iadd}\Slash{}\pyf{isub}\Slash{}\cdots(b)$ \` supports $a\verb.+=.b$, etc. \\
\'$\pyf{pow}(y, \optional{z})$ \` supporting $\pyf{pow}$ builtin function \\
\'$\pyf{neg}\Slash{}\pyf{pos}\Slash{}\pyf{abs}\Slash{}\pyf{invert}()$ \` supports unary \\ \` operators \verb.-., \verb.+., \pyf{abs}(), \verb.~. \\
\'$\pyf{complex}\Slash{}\pyf{int}\Slash{}\pyf{long}\Slash{}\pyf{float}()$ \` obj as number \\
\'$\pyf{oct}\Slash{}\pyf{hex}()$ \` string octal/hexidecimal repr
\pyv{25}{\\ \'$\pyf{index}()$ integer, if used as index }
\\ \'$\pyf{coerce}(b)$ \` $(a,b)$ as common num. type 
\pyv{25}{\\ \thead{Implement Context Manager} }
\pyv{25}{\'$\pyf{enter}()$ \` enter$\!$ context,$\!$ \pys{as} target$\!$ gets$\!$ retval }
\pyv{25}{\\ \'$\pyf{exit}(exc\_type, exc\_value, traceback)$ }
\pyv{25}{\\ \`args not \pys{None} when exception raised }
\pyv{25}{\\ \`\pys{True} retval suppresses; never reraise }
\end{tabbing}

\heading{Built-in Functions}

%, parsep=1pt

\begin{tabbing}
\hspace{2em}\= \kill
$\pyf{abs}(x)$ \` absolute value of $x$ \\
\pyv{25}{$\pyf{all}(it)$ \` every $x$ in $it$ has $\pyf{bool}(x) \eq \pys{True}$ \\}
\pyv{25}{$\pyf{any}(it)$ \` any $x$ in $it$ has $\pyf{bool}(x) \eq \pys{True}$ \\}
\pyv{26}{\'$\pyf{bin}(x)$ \` format number as binary \\}
\'$\pyf{callable}(obj)$ \` if $obj$ callable \\
$\pyf{chr}(i)$ \` ASCII to character \\
$\pyf{classmethod}(func)$ \` first arg to decorated \\ \` method is class instead of instance \\
\'$\pyf{cmp}(x,y)$ \` negative if $x<y$ \\
\'$\pyf{compile}(src, fname, mode, \optional{..})$ \\
%\` Produce a code object from source. \\
\> $src$ \` source code to compile \\
\> $fname$ \` fake filename \\
\> $mode$ \` $\pys{'[exec|eval|single]'}$ if $src$ \\
\` block/expression/interactive \\
%from $src$ as if read from \\ \` $fname$, $mode=\pys{'[exec|eval|single]'}$ if $src$ block/expression/interactive. \\
$\pyf{delattr}(obj, a)$ \` like $\pyk{del}\ obj.a$ \\
$\pyf{dir}()$ \` current scope's variable names \\
$\pyf{dir}(obj)$ \` $obj$'s attribute names \\
$\pyf{divmod}(a,b)$ \` (quotient, remainder) of $\frac{a}{b}$ \\
\pyv{26}{$\pyf{enumerate}(it, \optional{s\!\!=\!\!0})$ \` yield $(index+s, x)$} \pyV{26}{\'$\pyf{enumerate}(it)$ \` yield $(index, x)$ for $x$ in $it$} \\
\'$\pyf{eval}(expr, \optional{glob, \optional{loc}})$ \` interpret Python \\ \` expression with scope variables \\
\'$\pyf{execfile}(fname, \optional{glob, \optional{loc}})$ \\
\'$\pyf{file}( \ldots )$ \` \pyf{open} preferred \\
\'$\pyf{filter}(func, it)$ \` $x$ in $it$ with true $func(x)$ \\
\pyv{26}{\'$\pyf{format}(val, \optional{fspec})$ \` format $val$ by \\ \` format specification mini-language \\}
\'$\pyf{getattr}(obj,\!a,\!\optional{\mathop{de\!f}})$ \` obj.a if exists, or $de\!f$ \\
\'$\pyf{globals}()$ \` name-value dict of global vars \\
\'$\pyf{hasattr}(obj, a)$ \` if $obj.a$ exists \\
\'$\pyf{hash}(obj)$ \` hashcode of $obj$ \\
\'$\pyf{help}(\optional{obj})$ \` launch pydoc \\
\'$\pyf{hex}(x)$ \` format number as hexidecimal \\
\'$\pyf{id}(obj)$ \` object identifier (C pointer val) \\
\'$\pyf{input}(\optional{prompt})$ \` evaluate stdin input \\ %\\ \` equiv to $\pyf{eval}(\pyf{raw\_input}(prompt))$ \\
%\'$\pyf{int}(\optional{x, \optional{base}})$ \` foo \\
\'$\pyf{isinstance}(obj, cls)$ \` $obj$ instance of $cls$ \\
\'$\pyf{issubclass}(cls, sup)$ \` $cls$ subclass of $sup$ \\
\'$\pyf{iter}(obj)$ \` iterator over iterable $obj$ \\
\`over items (seqs) keys (maps) lines (files) \\
\'$\pyf{iter}(f, s)$ \` yield $f()$ till $s$ returned \\
\'$\pyf{len}(s)$ \` sequence length \\
\'$\pyf{locals}()$ \` name-value dict of local vars \\
%\'$\pyf{long}(\optional{x, \optional{base}})$ \` foo \\
\'$\pyf{map}(func, it, ...)$ \` list of $func(x, ...)$ \\
\pyv{25}{\'$\pyf{max}(it, \optional{key})$ \` $x$ in $it$ with $x$/$key(x)$ max \\}
\pyv{25}{\'$\pyf{max}(arg1, arg2, ..., \optional{key})$ \\ \` the arg with $arg$/$key(arg)$ max \\}
\pyV{25}{\'$\pyf{max}(it)$ \` maximum item in $it$ \\}
\pyV{25}{\'$\pyf{max}(arg1, arg2, ...)$ \` max of all args \\}
%\'$\pyf{memoryview}(obj)$ \` foo \\
\'$\pyf{min}(...)$ \` analogous to \pyf{max}, but minimum \\
\'$\pyf{next}(it, \optional{def})$ \` next item, or $def$ if done \\
%\'$\pyf{object}()$ \` foo \\
\'$\pyf{oct}(x)$ \` format number as octal \\
\'$\pyf{open}(fname, \optional{mode, \optional{bufsize}})$ \\
\> $fname$ \` file name \\
\> $mode$ \` \verb.rwab+U. (read, write, append, \\ \` binary, read+write, univ. newline) \\
\> $bufsize$ \` $0$ none, $1$ line buf., $>\!\!\!1$ size \\
\'$\pyf{ord}(c)$ \` ASCII code for character \\
\'$\pyf{pow}(x, y, \optional{z})$ \` ${x^y \bmod z}$ \\
\pyv{26}{\'$\pyf{print}(\optional{obj, \ldots}, \optional{sep=\pys{'\ '}},$ \\ \> $\optional{end=\pys{'\backslash n'}}, \optional{file=sys.stdout})$ \\ \` write $sep$-delimited $obj$s, $end$, to $file$ \\}
\'$\pyf{property}(\optional{fget, \optional{fset, \optional{fdel, \optional{doc}}}})$ \\ \` as member $a$ of $obj$, ops on $obj.a$ \\ \` handled by appropriate method \\
\'$\pyf{range}(\optional{b}, e, \optional{s})$ \` list $[b,\!b\!+\!s,\!...]$ to $e$ noninc. \\ %\` up to $e$ noninclusive \\
\'$\pyf{raw\_input}(\optional{prompt})$ \` terminal input \\
\'$\pyf{reduce}(func, it, \optional{init})$ \` $func$ $1^{\mathrm{st}}$ arg $init$ \\
\'$\pyf{reload}(module)$ \` re-read module code \\
\'$\pyf{repr}(obj)$ \` eval-able string for $obj$ \\
\pyv{24}{\'$\pyf{reversed}(seq)$ \` reverse iterator \\}
\'$\pyf{round}(x, \optional{n\!=\!0})$ \` round $x$ to $n$ places \\
\'$\pyf{setattr}(obj, a, v)$ \` like $obj.a=v$ \\
%\'$\pyf{slice}(\optional{b}, e, \optional{s})$ \`  \\
\'$\pyf{sorted}(it, \optional{cmp, \optional{key, \optional{reverse}}})$ \\ \` sorted list of items $x$ in $it$ \\
\> $cmp$ \` define ordering like \pyf{cmp} \\
\> $key$ \` order $key(x)$ instead of $x$ in $it$ \\
\> $reverse$ \` if true, decending order \\
\'$\pyf{staticmethod}(func)$ \` first arg to \\ \` decorated method is not instance \\
%\'$\pyf{str}(\optional{obj})$ \` foo \\
\'$\pyf{sum}(it, \optional{s=0})$ \` sum of $x$ in $it$, plus $s$ \\
\'$\pyf{super}(type, \optional{obj\_or\_type})$ \` proxy object \\ \` delegates calls to parent/sibling of $type$ \\
%\'$\pyf{tuple}(\optional{iterable})$ \` foo \\
\'$\pyf{type}(obj)$ \` get $obj$'s type \\
\'$\pyf{type}(name, bases, dict)$ \` make new type \\
\'$\pyf{unichr}(i)$ \` unicode character given  \\
%\'$\pyf{unicode}(\optional{obj, \optional{encoding, \optional{errors}}})$ \` foo \\
\'$\pyf{vars}(\optional{obj})$ \` $obj.\pysec{dict}$ if $obj$ else $\pyf{locals}()$ \\
\'$\pyf{xrange}(\optional{b}, e, \optional{s})$ \` nonmaterialized \pyf{range} \\
\'$\pyf{zip}(\optional{it, ...})$ \` list of tuples, $i^\mathrm{th}$ tuple \\ \` contains each iterator's $i^\mathrm{th}$ item
\end{tabbing}

\heading{Regular Expressions, \pyp{re}}

Module has these flag constants and functions.  In these, $p$ is a pattern, $s$ is the string we search, $f$ are or-ed flags, $c$ is number of ops.
\begin{tabbing}
\hspace{2em}\= \kill
\'$\pys{IGNORECASE}/\pys{I}$ \` case insensitive-matching \\
\'$\pys{LOCALE}/\pys{L}$ \` \verb.\wWbBsS. locale dependent \\
\'$\pys{MULTILINE}/\pys{M}$ \` \verb.^$. match begin/end lines \\
\'$\pys{DOTALL}/\pys{S}$ \` \verb#.# matches \verb.\n. \\
\'$\pys{UNICODE}/\pys{U}$ \` Uni. charprop for \verb.\wWbBdDsS. \\
\'$\pys{VERBOSE}/\pys{X}$ \` space ignored, \verb.#. comments \\
\'$\pyf{compile}(p, \optional{f})$ \` \pyt{RegexObject} on $p$, flags $f$ \\
\'$\pyf{search}(p, s, \optional{f})$ \` find match in $s$ \\
\'$\pyf{match}(p, s, \optional{f})$ \` match start of $s$ \\
\'$\pyf{split}(p, s, \optional{c\pyv{27}{, \optional{f}}})$ \` $p$ splits $s$ up to $c$ times \\
\` capturing groups in $p$ included in list \\
\'$\pyf{findall}(p, s\pyv{24}{, \optional{f}})$ \` $s$ substrs matching $p$ \\
\` if one group in $p$, item is group string \\
\` if multiple groups, item is tuple of groups \\
\'$\pyf{finditer}(p, s\pyv{24}{, \optional{f}})$ \` yields \pyt{MatchObject}s \\
\'$\pyf{sub}(p, r, s, \optional{c\pyv{27}{, \optional{f}}})$ \` matches$\!$ replaced$\!$ with$\!$ $r$ \\
\` if$\!$ $r$ func,$\!$ gets$\!$ \pyt{MatchObject},$\!$ returns$\!$ string \\
\'$\pyf{subn}(p, r, s, \optional{c\pyv{27}{, \optional{f}}})$ \` same,$\!$ but$\!$ returns$\!$ tuple \\ \` \verb.(replaced_string, numsubs). \\
\'$\pyf{escape}(s)$ \` escape out RE special strings \\
\thead{\pyt{RegexObject} Methods/Attrs}
\'$\pyf{search}\Slash\pyf{match}\Slash\pyf{findall}\Slash\pyf{finditer}(s,\!\optional{b,\!\optional{e}})$ \` \\
\'$\pyf{sub}\Slash\pyf{subn}(r, s, \optional{c})$ \` similar to module funcs \\
\'$\pyf{split}(s, \optional{c})$ \` but compiled, and $b,\!e$ substr \\
\'$\pys{flags}$ \` which flags it was compiled with \\
\'$\pys{groups}$ \` number of capturing groups \\
\'$\pys{groupindex}$ \` \verb.(?P<id>). names to nums \\
\'$\pys{pattern}$ \` the pattern that was compiled \\
\thead{\pyt{MatchObject} Methods/Attrs}
\'$\pyf{expand}(t)$ \` backslash$\!$ subs$\!$ on$\!$ $t\!$ using$\!$ match \\
\'$\pyf{group}(\optional{group1, ...})$ \` get indicated group(s) \\
\'$\pyf{groups}(\optional{de\!f\!=\!\pys{None}})$ \` get all groups, \\ \` replacing empties with $de\!f$ \\
\'$\pyf{groupdict}(\optional{de\!f})$ \` map$\!$ named$\!$ groups$\!$ to$\!$ grp \\
\'$\pyf{start}(\optional{grp})$ \` start index of match/group \\
\'$\pyf{end}(\optional{grp})$ \` end index of match/group \\
\'$\pyf{span}(\optional{grp})$ \` tuple of start, end \\
\'$\pyf{pos}$ \` $b$ passed to \pyt{RegexObject} method \\
\'$\pyf{endpos}$ \` $e$ passed to \pyt{RegexObject} method \\
\'$\pyf{lastindex}$ \` index of last matched group \\
\'$\pyf{lastgroup}$ \` name of last matched group \\
\'$\pyf{re}$ \` \pyt{RegexObject} that produced this \\
\'$\pyf{string}$ \` string this match came from \\
\end{tabbing}

\subheading{RE Pattern Syntax}
\begin{multicols}{2}
\begin{tabbing}
\hspace{2em}\= \kill
\' \verb#.# \` any char but \verb.\n. \\ 
\' \verb#^# \` match str start \\ 
\' \verb#$# \` match str end \\ 
\' \verb#\# \` escape spec. chars \\
\' \verb#\#$n$ \` match group $n$ \\
\' \verb#\A# \` match str start \\
\' \verb#\b# \` word boundary \\
\' \verb#\d# \` mass decimal \\
\' \verb#\s# \` whitespace char \\
\' \verb#\w# \` alphanum or \_ \\
\' \verb#\Z# \` match str end \\
\end{tabbing}
\end{multicols}
\vspace{-1.6em}
\begin{tabbing}
\hspace{2em}\= \kill
\' \verb#\B#,\verb#\D#,\verb#\S#,\verb#\W# \` compliment of \verb#\b#,\verb#\d#,\verb#\s#,\verb#\w# \\
\' \verb#\a#,\verb#\b#,\verb#\f#,\verb#\n#,\verb#\r#,\verb#\t#,\verb#\v#,\verb#\x#,\verb#\\# \` reg. escapes \\
\' \verb#?# \` match $0,1$ reps of preceding \\
\' \verb#*# \` match $\geq 0$ reps of preceding \\
\' \verb#+# \` match $\geq 1$ reps of preceding \\
\' \verb#{m}# \` match $m$ reps of preceding \\
\' \verb#{m,n}# \` match $m$ to $n$ reps of preceding \\
\' \verb#*?#, \verb#+?#, \verb#??#, \verb#{m,n}?# \` non-greedy variants \\
\' \verb#|# \` for $A|B$ match $A$ or $B$ \\
\' \verb#[#$chars$\verb#]# \` match set of chars \\
\' \verb#[^#$chars$\verb#]# \` match anything but chars \\
\' \verb#(...)# \` beginning and end of group \\
\' \verb#(?iLmsux)# \` set$\!$ flags$\!$ \verb.ILMSUX. in$\!$ the$\!$ pattern \\
\' \verb#(?:...)# \` non-grouping regular parens \\
\' \verb#(?P<#$n$\verb#>...)# \` named group $n$ \\
\' \verb#(?P=#$n$\verb#)# \` match previously named group $n$ \\
\' \verb-(?#...)- \` a comment, contents ignored \\
\' \verb#(?=...)# \` lookahead,$\!$ match,$\!$ don't$\!$ consume \\
\' \verb#(?!...)# \` negative lookahead assertion \\
\' \verb#(?<=...)# \` positive lookbehind assertion \\
\' \verb#(?<!...)# \` negative lookbehind assertion \\
\' \verb#(?(#$id\!/\!name$\verb#)#$y$\verb#|#$n$\verb#)# \\ \` match $y$ if group $id/name$ exists, else $n$
\end{tabbing}
\heading{\pyp{datetime}}

\begin{tabbing}
\hspace{2em}\= \kill
\'$\pys{MINYEAR}$\` smallest allowed year, 1 \\
\'$\pys{MAXYEAR}$\` largest allowed year, 9999 \\
\thead{Shared Methods/Members}
\'$\pys{\class.min}$ \` most negative/earliest instance \\
\'$\pys{\class.max}$ \` most positive/earliest instance \\
\'$\pys{\class.resolution}$ \` smallest possible difference \\
\'$\pys{initarg}$ \` init args usually members, e.g., \\ \` \pyt{date} have \pys{year}, \pys{month}, \pys{day} members \\
\thead{\pyt{timedelta}}
\'$\pyf{\class}(days, seconds, microseconds,$\\ \`$milliseconds, minutes, hours, weeks)$\\ \`all args optional, can be float, default 0 \\ \` only first three args become members \\
\pyv{27}{\'$\pyf{total\_seconds}()$ \` total seconds in delta \\}
\thead{\pyt{datetime}}
\'$\pyf{\class}(year, month, day, {[}hour, minute, second, $\\
\>$microsecond,\! tzin\!f\!o{]})$ \` date$\!$/$\!$time comb. \\
\'$\pyf{\class.today}()$ \` current local date/time \\
\'$\pyf{\class.now}(\optional{tz})$ \` similar, but with \pyt{tzinfo} \\
\'$\pyf{\class.utcnow}()$ \` current UTC date/time \\
\'$\pyf{\class.fromtimestamp}(ts, \optional{tz})$ \\ \` local date/time from POSIX timestamp \\
\'$\pyf{\class.utcfromtimestamp}(ts)$ \\ \` UTC date/time from POSIX timestamp \\
\'$\pyf{\class.fromordinal}(ord)$ \` $ord\eq 1$ is 1-Jan-1 \\
\'$\pyf{\class.combine}(d, t)$ \` mix \pyt{date} and \pyt{time} inst. \\
\'$\pyf{\class.strptime}(s, f)$ \` parse $s$ formatted as $f$ \\
\'$\pyf{date}()$ \` \pyt{date} instance with the same date \\
\'$\pyf{time}()$ \` \pyt{time} inst. time, \pys{tzinfo}\eq\pys{None} \\
\'$\pyf{timetz}()$ \` \pyt{time} inst. with same \pys{tzinfo} \\
\'$\pyf{replace}(year, month, day, hour, minute,$\\
\> $second, microsecond, tzin\!f\!o)$ \\ \` new instance, indicated fields replaced \\
\'$\pyf{astimezone}(tz)$ \` adjusted to timezome \\
\'$\pyf{utcoffset}()$/$\pyf{dst}()$/$\pyf{tzname}()$ \\ \` calls same method on \pys{tzinfo} with \pys{self} \\
\'$\pyf{timetuple}()$ \` convert$\!$ to$\!$ \pyp{time}\pyt{.struct\_time} \\
\'$\pyf{utctimetuple}()$ \` tzone adjusted to UTC \\
\'$\pyf{toordinal}()$ \` day num if 1-Jan-1 is day 1 \\
\'$\pyf{weekday}()$ \` weekday, Monday$\eq 0$, etc. \\
\'$\pyf{isoweekday}()$ \` weekday, Monday$\eq 1$, etc. \\
\'$\pyf{isocalendar}()$ \` ISO {\small year,weeknum,weekday} \\
\'$\pyf{isoformat}(\optional{sep})$ \` msecs if $\!\neq\!0$, UTC offset \\ \` \texttt{\small{}YYYY$\!$-MM-DD[$\!sep\!$]HH$\!$:$\!$MM$\!$:$\!$SS[$\!$.mmmmmm]$\!$[+HH$\!$:$\!$MM]} \\
\'$\pyf{ctime}()$ \` e.g., \verb.Tue Jun 22 21:45:35 2010. \\
\'$\pyf{strftime}(f)$ \` string formatted by $f$ \\
\thead{\pyt{date}}
\'$\pyf{\class}(year, month, day)$ \` init with date \\
\'$\pyf{\class.today}()$ \` current local date \\
\'$\pyf{\class.fromtimestamp}(ts)$ \` from POSIX t.s. \\
\'$\pyf{\class.fromordinal}(ord)$ \` $ord\eq 1$ is 1-Jan-1 \\
%\'$\pys{year}$ \` in range $[\pys{MINYEAR}, \pys{MAXYEAR}]$ \\
%\'$\pys{month}$ \` in range $[1, 12]$ \\
%\'$\pys{day}$ \` in range $1$ to month's days \\
\'$\pyf{replace}(year, month, day)$ \` get repl. date \\
\'$\pyf{timetuple}()$ \` convert$\!$ to$\!$ \pyp{time}\pyt{.struct\_time} \\
\'$\pyf{toordinal}()$ \` day num if 1-Jan-1 is day 1 \\
\'$\pyf{weekday}()$ \` weekday, Monday$\eq 0$, etc. \\
\'$\pyf{isoweekday}()$ \` weekday, Monday$\eq 1$, etc. \\
\'$\pyf{isocalendar}()$ \` ISO {\small year,weeknum,weekday} \\
\'$\pyf{isoformat}()$ \` in format \verb.YYYY-MM-DD. \\
\'$\pyf{ctime}()$ \` e.g., \verb.Tue Jun 22 00:00:00 2010. \\
\'$\pyf{strftime}(f)$ \` date string formatted by $f$ \\
\thead{\pyt{time}}
\'$\class(\!hour\!,\!minute,\!second,\!micr\!osecond,\!tzin\!f\!o\!)$ \\
\` all args optional, default to 0/\pys{None} \\
\'$\pyf{replace}(\!hour\!,\!minute,\!second,\!micr\!osecond,$\\
\>$tzin\!f\!o)$ \` new inst. with fields replaced \\
\'$\pyf{isoformat}()$ \` msecs if $\!\neq\!0$, UTC offset if tz \\ \` \texttt{HH$\!$:$\!$MM$\!$:$\!$SS[$\!$.mmmmmm]$\!$[+HH$\!$:$\!$MM]} \\
\'$\pyf{strftime}(f)$ \` time string formatted by $f$ \\
\'$\pyf{utcoffset}()$/$\pyf{dst}()$/$\pyf{tzname}()$ \\ \` calls same method on \pys{tzinfo} with \pys{self} \\
\thead{\pyt{tzinfo}}
\'$\pyf{utcoffset}(dt)$ \` delta from UTC inc. DST \\
\'$\pyf{dst}(dt)$ \` DST timedelta offset ($\!$e.g. 0$\!$/$\!$1 hrs$\!$) \\
\'$\pyf{tzname}(dt)$ \` description of timezone \\
\'$\pyf{fromutc}(dt)$ \` UTC time to local time \\ \` subclasses rarely override default
\end{tabbing}

\subheading{\pyf{strftime}/\pyf{strptime} Format Ops}
\begin{multicols}{2}
\begin{tabbing}
\hspace{2em}\= \kill
\' \verb#a# \` abr wkday name \\
\' \verb#A# \` full wkday name \\
\' \verb#b# \` abr month name \\
\' \verb#B# \` full month name \\
\' \verb#c# \` date \& time repr \\
\' \verb#d# \` day as \verb#01#-\verb#31# \\
\' \verb#f# \` zero padded $\mu$sec \\
\' \verb#H# \` hour as \verb#00#-\verb#23# \\
\' \verb#I# \` hour as \verb#01#-\verb#12# \\
\' \verb#j# \` yr day as \verb#001#-\verb#366# \\
\' \verb#m# \` month as \verb#01#-\verb#12# \\
\' \verb#M# \` min as \verb#00#-\verb#59# \\
\' \verb#p# \` \verb#AM#, \verb#PM# \\
\' \verb#S# \` sec as \verb#00#-\verb#61# \\
\' \verb#U# \` week num \verb#00#-\verb#53# \\
\` wk 0 before 1$\!^{st}$ Su \\
\' \verb#w# \` wkdy $\!$a$\!$s$\!$ \verb#0#-\verb#6# ($\!$Su-S$\!$a$\!$) \\
\' \verb#W# \` $\sim$\verb#U#, but from Mo \\
\' \verb#x# \` date repr \\
\' \verb#X# \` time repr \\
\' \verb#y# \` year as $00-99$ \\
\' \verb#Y# \` year with century \\
\' \verb#z# \` UTC $\!$off$\!$.$\!$ as {\small $\pm\!\mathtt{HHMM}$} \\
\' \verb#Z# \` timezone name
%\' \verb#%# \`  
\end{tabbing}
\end{multicols}
\pyv{24}{\heading{\pyp{collections}}

\begin{tabbing}
\hspace{2em}\= \kill
\pyV{27}{\begin{comment}}
\thead{\pyt{Counter}}
\parbox{\linewidth}{Maps elements to count, like a multiset.  Is \pyt{dict} subclass, but \pyf{fromkeys} inapplicable.} \\
\' $\class()$ \` empty counter \\
\' $\class(it)$ \` count elements in $it$ \\
\' $\class(map)$ \` element maps to counts \\
\'$\pyf{elements}()$ \` iter, elems repeat count times  \\
\'$\pyf{most\_common}(\optional{n})$ \` $n$ top elem/count pairs \\
\'$\pyf{update}(\optional{it\_or\_map})$ \` increments counts \\
\'$\pyf{subtract}(\optional{it\_or\_map})$ \` decrements counts \\
\' $\insta$\verb#+#$\instb$ \` add counters together \\
\' $\insta$\verb#-#$\instb$ \` subtract, keeps only positives \\
\' $\insta$\verb#&#$\instb$ \` intersect, keeps minimum count \\
\' $\insta$\verb#|#$\instb$ \` union, keeps maximum count \\
\pyV{27}{\end{comment}}
\pyV{24}{\begin{comment}}
\thead{\pyt{deque}}
\parbox{\linewidth}{Generalizes stacks and queues.}\\
\'$\pyf{\class}(\optional{it\pyv{26}{,\!\optional{maxlen}}})$ \` inits with $it$'s items \\ \pyv{26}{\` len capped at $maxlen$, items \\ \` discarded from opposite end \\}
\'$\pyf{append}(x)$ \` add $x$ to right side \\
\'$\pyf{clear}()$ \` empty the deque \\
\pyv{27}{\'$\pyf{count}(x)$ \` count items equal to $x$ \\}
\'$\pyf{extend}(it)$ \` add $it$'s items to right side \\
\'$\pyf{pop}()$ \` remove, return rightmost item  \\
\'$\pyf{appendleft}$/$\pyf{extendleft}$/$\pyf{popleft}$ \\ \` similar, but ops on the left side \\
\pyv{25}{\'$\pyf{remove}(x)$ \` remove first $x$, or \pyt{ValueError} \\}
\'$\pyf{reverse}()$ \` reverse element order in place \\
\'$\pyf{rotate}(n)$ \` rotate $n$ step right (left if neg)
\pyv{27}{\\ \'$\pys{maxlen}$ \` max deque size, \pys{None} if unbound } \pyV{24}{\end{comment}}\pyV{25}{\begin{comment}}
\\ \thead{\pyt{defaultdict}}
\' $\class(\!\optional{\!f\!act,\!\optional{..}})$ \` like \pyt{\!d\!i\!c\!t\!};$\!$ missing keys get $\!\!f\!act\!()$ \\
\' \pys{default\_factory} \` callable$\!$ for$\!$ default$\!$ vals
\pyV{25}{\end{comment}}
\pyV{26}{\begin{comment}}
\\ \thead{\pyt{namedtuple}}
\parbox{\linewidth}{Fixed len tuple type with named fields.}\\
\' $\class(name, f\!ields, \optional{verbose, \optional{rename}})$ \\
\> $name$ \` the type's name \\
\> $f\!ields$ \` space$\!$/$\!$comma delim \pyt{str}, or seq \\
\> $verbose$ \` if $\pys{True}$, prints class def \\
\> $rename$ \` bad fieldnames replaced with \\ \` \verb#_#$d$ positional names for index $d$ \\
\'$\pyf{\_make}(it)$ \` make instance from sequence \\
\'$\pyf{\_asdict}()$ \` map of field names to values \\
\'$\pyf{\_replace}(kwargs)$ \` copy tuple, use k.w. \\ \` args to replace values \\
\'$\pyf{\_fields}$ \` the tuple of string fieldnames
\pyV{26}{\end{comment}}
\pyV{27}{\begin{comment}}
\\ \thead{\pyt{OrderedDict}}
\parbox{\linewidth}{Like \pyt{dict}, but remembers insertion order.}\\
\' $\class(..)$ \` acts like \pyt{dict} constructor \\
\'$\pyf{popitem}(last\verb#=#\pys{True})$ \` remove and return \\ \` key/value pair, LIFO if $last$ else FIFO
\pyV{27}{\end{comment}}
\pyV{26}{\begin{comment}}
\end{tabbing}

\subheading{Abstract Base Classes}
Subclasses implement $\mathbb{A}$bstract methods, $\mathbb{M}$ixin methods provided.
\begin{tabbing}
\hspace{2em}\= \kill
\' \pyt{Container} \` $\mathbb{A}$:\pysec{contains} \\
\' \pyt{Hashable} \` $\mathbb{A}$:\pysec{hash} \\
\' \pyt{Iterable} \` $\mathbb{A}$:\pysec{iter} \\
\' \pyt{Iterator}$(\!\pyt{Iterable}\!)$ \` $\mathbb{A}$:\pyf{next}, $\mathbb{M}$:\pysec{iter} \\
\' \pyt{Sized} \` $\mathbb{A}$:\pysec{len} \\
\' \pyt{Callable} \` $\mathbb{A}$:\pysec{call} \\
\' \pyt{Sequence}$(\pyt{Sized},\pyt{Iterable},\pyt{Container})$ \\ \` $\mathbb{A}$:\pysec{getitem}, $\mathbb{M}$:\pysec{contains}, \\
\` \pysec{iter}, \pysec{reversed}, \pyf{index}, \pyf{count} \\
\' \pyt{MutableSequence}$(\pyt{Sequence})$ \\ \` $\mathbb{A}$:\pysec{setitem}, \pysec{delitem}, \pyf{insert} \\
\` $\mathbb{M}:$\pyf{append}, \pyf{reverse}, \pyf{extend}, \\
\` \pyf{pop}, \pyf{remove}, \pysec{iadd} \\
\' \pyt{Set}$(\pyt{Sized},\pyt{Iterable},\pyt{Container})$ \\ \` $\mathbb{M}:$\pysec{le\!/\!lt\!/\!eq\!/\!ne\!/\!gt\!/\!ge\!/\!and\!/\!or\!/\!sub\!/\!xor} \\
\' \pyt{MutableSet}$(\pyt{Set})$ \` $\mathbb{A}:$\pyf{add}, \pyf{discard} \\
\' \pyt{Mapping}$(\pyt{Sized},\pyt{Iterable},\pyt{Container})$ \\
\` $\mathbb{A}:$\pysec{getitem}, $\mathbb{M}$:\pysec{contains}, \pyf{keys}, \\
\` \pyf{items}, \pyf{values}, \pyf{get}, \pysec{eq}, \pysec{ne} \\
\' \pyt{MutableMapping}$(\pyt{Mapping})$ \\ \` $\mathbb{A}:$\pysec{setitem}, \pysec{delitem}, $\mathbb{M}:$\pyf{pop}, \\ \` \pyf{popitem}, \pyf{clear}, \pyf{update}, \pyf{setdefault} \\
\' \pyt{MappingView}$(\pyt{Sized})$ \` $\mathbb{M}:$\pysec{len} \\
\' \pyt{KeysView}$(\pyt{MappingView},\pyt{Set})$ \\ \` $\mathbb{M}:$\pysec{contains}, \pysec{iter} \\
\' \pyt{ItemsView}$(\pyt{MappingView},\pyt{Set})$ \` \emph{same} \\
\' \pyt{ValuesView}$(\pyt{MappingView})$ \` \emph{same} \\
\end{tabbing}
\pyV{26}{\end{comment}}
}
\heading{\pyp{heapq}}

Heap queue algorithm.  Many functions operate on a heapified list $h$.

\begin{tabbing}
\hspace{2em}\= \kill
\'$\pyf{heappush}(h, v)$ \` add $v$ to heap
\\ \'$\pyf{heappop}(h)$ \` remove, return smallest
\pyv{26}{\\ \'$\pyf{heappushpop}(h, v)$ \` like push then pop}
\\ \'$\pyf{heapreplace}(h, v)$ \` like pop then push
\\ \'$\pyf{heapify}(x)$ \` make list $x$ into heap
\pyv{26}{\\ \'$\pyf{merge}(*x)$ \` given sorted inputs, single \\ \` iterator over merged sorted items}
\pyv{24}{\\ \'$\pyf{nlargest}(n, it\pyv{25}{, \optional{key}})$ \` top $\!n\!$ items from $it$\pyv{25}{, \\ \` or with top values from $key(x)$}}
\pyv{24}{\\ \'$\pyf{nsmallest}(n, it\pyv{25}{, \optional{key}})$ \` same, but $n$ least}
\end{tabbing}
\heading{\pyp{bisect}}

Binary search for $x$ on sorted seqeucen $s$.  Can operate on $b\!:\!e$ sublist instead.

\begin{tabbing}
\hspace{2em}\= \kill
\'$\pyf{bisect\_left}(s,\!x,\!\optional{b,\!\optional{e}})$ \` leftmost index to \\ \` insert $x$ in $s$ to keep $s$ sorted \\
\'$\pyf{bisect\_right}(s,\!x,\!\optional{b,\!\optional{e}})$ \` rightmost, same \\
\'$\pyf{bisect}(...)$ \` same as \pyf{bisect\_right} \\
\'$\pyf{insort\_left}(s,\!x,\!\optional{b,\!\optional{e}})$ \` same, but does \\
\'$\pyf{insort\_right}(s,\!x,\!\optional{b,\!\optional{e}})$ \` the insertion \\
\'$\pyf{insort}(...)$ \` same as \pyf{insort\_right} \\
\end{tabbing}
\heading{\pyp{os}}

Operating system dependent module.

\begin{tabbing}
\hspace{2em}\= \kill
\'\pys{name} \\
\thead{Process Parameters}
\'$\pyf{environ}$ \`  \\
\'$\pyf{chdir}(path)$ \`  \\
\'$\pyf{fchdir}(fd)$ \`  \\
\'$\pyf{getcwd}()$ \`  \\
\'$\pyf{ctermid}()$ \`  \\
\'$\pyf{getegid}()$ \`  \\
\'$\pyf{geteuid}()$ \`  \\
\'$\pyf{getgid}()$ \`  \\
\'$\pyf{getgroups}()$ \`  \\
\'$\pyf{initgroups}(username, gid)$ \`  \\
\'$\pyf{getlogin}()$ \`  \\
\'$\pyf{getpgrp}()$ \`  \\
\'$\pyf{getpid}()$ \`  \\
\'$\pyf{getppid}()$ \`  \\
\'$\pyf{getresuid}()$ \`  \\
\'$\pyf{getresgid}()$ \`  \\
\'$\pyf{getuid}()$ \`  \\
\'$\pyf{getenv}(varname, \optional{value})$ \`  \\
\'$\pyf{putenv}(varname, value)$ \`  \\
\'$\pyf{setegid}(egid)$ \`  \\
\'$\pyf{seteuid}(euid)$ \`  \\
\'$\pyf{setgid}(gid)$ \`  \\
\'$\pyf{setgroups}(groups)$ \`  \\
\'$\pyf{setpgrp}()$ \`  \\
\'$\pyf{^g\!setpgid}(pid, pgrp)$ \`  \\
\'$\pyf{setregid}(rgid, egid)$ \`  \\
\'$\pyf{setresgid}(rgid, egid, sgid)$ \`  \\
\'$\pyf{setresuid}(ruid, euid, suid)$ \`  \\
\'$\pyf{setreuid}(ruid, euid)$ \`  \\
\'$\pyf{getsid}(pid)$ \`  \\
\'$\pyf{setsid}()$ \`  \\
\'$\pyf{setuid}(uid)$ \`  \\
\'$\pyf{strerror}(code)$ \`  \\
\'$\pyf{umask}(mask)$ \`  \\
\'$\pyf{uname}()$ \`  \\
\'$\pyf{unsetenv}(varname)$ \`  \\
\end{tabbing}

\end{multicols}

\end{document}
